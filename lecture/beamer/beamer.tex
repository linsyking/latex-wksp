\documentclass[aspectratio=169]{beamer}
\usepackage{listings}
\usepackage{graphicx}
\usepackage{hyperref}

\newenvironment{fragileframe}
{\begin{frame}[fragile,environment=fragileframe]}
{\end{frame}}

\newenvironment{diy}[1]
{\begin{block}{\textbf{DIY #1}}}
{\end{block}}

\setbeamertemplate{navigation symbols}{}

\hypersetup{
    colorlinks=true,
    linkcolor=blue,
    filecolor=magenta,
    urlcolor=blue
}

\urlstyle{same}

\definecolor{codegreen}{rgb}{0,0.6,0}
\definecolor{codegray}{rgb}{0.5,0.5,0.5}
\definecolor{codepurple}{rgb}{0.58,0,0.82}
\definecolor{backcolour}{rgb}{0.95,0.95,0.92}

\lstdefinestyle{wksp}{
    language=[latex]tex, % format: [dialect]language
    backgroundcolor=\color{backcolour},
    commentstyle=\color{codegreen},
    keywordstyle=\color{magenta},
    numberstyle=\tiny\color{codegray},
    stringstyle=\color{codepurple},
    basicstyle=\ttfamily\footnotesize,
    breakatwhitespace=false,
    breaklines=true,
    captionpos=b,
    keepspaces=true,
    numbers=left,
    numbersep=5pt,
    showspaces=false,
    showstringspaces=false,
    showtabs=false,
    tabsize=4
}

\lstset{style=wksp}

\graphicspath{{./img/}}

\AtBeginSection[]{
    \begin{frame}{Table of Contents}
        \tableofcontents[currentsection]
    \end{frame}
}

\renewcommand{\tt}{\texttt}
\renewcommand{\bf}{\textbf}

\def\pre{1} % 1 to keep \pauses, 0 to skip

\title{\LaTeX\ workshop: Beamers}
\author{Frederick Yin}
\institute{TechJI}
\date{2023}

\begin{document}
\frame{\titlepage}

\section{Overview}
\begin{frame}{Why \tt{beamer}?}
\begin{columns}
    \begin{column}{0.4\textwidth}
        \begin{itemize}
            \item Content-centric
            \item Math typesetting
            \item Looks professional
            \item Reuse \LaTeX\ you wrote earlier
        \end{itemize}
    \end{column}
    \begin{column}{0.4\textwidth}
        \includegraphics[width=\textwidth]{file_extensions}
        \tiny{Randall Munroe, ``File Extensions'', \url{https://xkcd.com/1301/}}
    \end{column}
\end{columns}
\end{frame}

\begin{fragileframe}{Beamers 101}
Instead of \tt{article}, the first lines goes:
\begin{lstlisting}
\documentclass{beamer}
\end{lstlisting}
Then metadata:
\begin{lstlisting}
\title{\LaTeX\ workshop: Beamers}
\author{Frederick Yin}
\institute{TechJI}
\date{2023}
\end{lstlisting}
Then the document:
\begin{lstlisting}
\begin{document}
% frames
\end{document}
\end{lstlisting}
\if\pre1\pause\fi
\begin{diy}{1}
    In \tt{diy.tex}, fill in your name, then compile.
\end{diy}
\end{fragileframe}

\begin{fragileframe}{Frames}
A beamer is made of many \bf{frames}:
\begin{lstlisting}
\begin{frame}{Title}
    Frame content
\end{frame}
\end{lstlisting}
Or alternatively
\begin{lstlisting}
\begin{frame}
    \frametitle{Title}
    Frame content
\end{frame}
\end{lstlisting}
\end{fragileframe}

\section{Multicolumn}
\begin{fragileframe}{Multicolumn}
\begin{columns}
    \begin{column}{0.5\textwidth}
\begin{lstlisting}
\begin{columns}
    \begin{column}{0.5\textwidth}
        % ...
    \end{column}
    \begin{column}{0.3\textwidth}
        And that's how this frame
        was made!
    \end{column}
\end{columns}
\end{lstlisting}
    \end{column}
    \begin{column}{0.3\textwidth}
        And that's how this frame was made!
        \begin{block}{Note}
            This is a \tt{beamer} feature, \bf{not} the same thing as
            the \tt{multicols} environment in an article.
        \end{block}
    \end{column}
\end{columns}
\end{fragileframe}

\begin{fragileframe}{Alignment option}
\begin{columns}[t]
    \begin{column}{0.5\textwidth}
        Add option \tt{t} to top-align:
\begin{lstlisting}
\begin{columns}[t]
    \begin{column}{0.5\textwidth}
        % ...
    \end{column}
    \begin{column}{0.3\textwidth}
        % ...
    \end{column}
\end{columns}
\end{lstlisting}
    \end{column}
    \begin{column}{0.3\textwidth}
        All options:
        \begin{itemize}
            \item \tt{c} --- center
            \item \tt{t} --- top
            \item \tt{T} --- different top
                \footnote[frame]{See section 12.7 of \tt{beamer} docs.}
            \item \tt{b} --- bottom
        \end{itemize}
    \end{column}
\end{columns}
\if\pre1\pause\fi
\begin{diy}{2}
    Divide frame ``
    $\Box\Box\Box\Box,\Box\Box\langle\langle\Box\Box\rangle\rangle$
    '' into two columns.
\end{diy}
\end{fragileframe}

\section{Sections}
\begin{fragileframe}{Sections}
Use sections to outline your presentation.
\begin{lstlisting}
\section{}
% Frames
\subsection{}
% More frames
\end{lstlisting}
\begin{diy}{3}
    Group the lab report into sections, then compile and observe.
\end{diy}
\end{fragileframe}

\begin{fragileframe}{Table of Contents}
Insert this frame where you want a Table of Contents
(usually right after title page):
\begin{lstlisting}
\begin{frame}{Table of Contents}
    \tableofcontents
\end{frame}
\end{lstlisting}

\if\pre1\pause\fi
Or automatically insert one before each section:
\begin{lstlisting}
\AtBeginSection[]{
    \begin{frame}{Table of Contents}
        \tableofcontents[currentsection]
    \end{frame}
}
\end{lstlisting}
(In the preamble, before beginning of document)
\end{fragileframe}

\section{Blocks}
\begin{fragileframe}{Blocks}
Some block environments always need a title:
\begin{block}{Like this default block}
\begin{lstlisting}
\begin{block}{Like this default block}
\end{block}
\end{lstlisting}
\end{block}

\begin{alertblock}{Or this alertblock}
\begin{lstlisting}
\begin{alertblock}{Or this alertblock}
\end{alertblock}
\end{lstlisting}
\end{alertblock}

\end{fragileframe}

\begin{fragileframe}{Blocks}
Others have a default title, but you can elaborate:
\begin{example}
    This example does not tell you what it's about…
\begin{lstlisting}
\begin{example}
\end{example}
\end{lstlisting}
\end{example}

\begin{example}[with a description]
    But this one does!
\begin{lstlisting}
\begin{example}[with a description]
\end{example}
\end{lstlisting}
\end{example}
\end{fragileframe}

\begin{fragileframe}{Blocks}
Apart from \tt{example}, we have \tt{examples}, \tt{definition},
\tt{definitions}, \tt{theorem}, \tt{proof}, and \tt{corollary},
with similar syntax.
\begin{lstlisting}
\begin{something}[optional description]
    % Block content
\end{something}
\end{lstlisting}

\begin{diy}{4}
    Put each paragraph in a block you think is suitable.
\end{diy}
\end{fragileframe}

\section{Overlays}
\begin{fragileframe}{Pauses}
``Overlay'' is just a fancy way of saying ``hiding part of the frame''.
The simplest way is \verb|\pause|:
\begin{lstlisting}
\begin{frame}
    \begin{itemize}
        \item To be, or --- next slide, please ---
        \pause
        \item not to be, that is the question.
    \end{itemize}
\end{frame}
\end{lstlisting}
\if\pre1\pause\fi
\only<2>{Slide 1}
\only<3>{Slide 2}
\begin{itemize}
    \item To be, or --- next slide, please ---
    \pause
    \item not to be, that is the question.
\end{itemize}
\if\pre1\pause\fi
\bf{Problem: not very customizable}
\end{fragileframe}

\begin{fragileframe}{Overlay specifications}
\bf{Solution: ``overlay specifications'' (aka angle brackets)}
\onslide<2->
\begin{lstlisting}
\begin{frame}
    \begin{itemize}
        \item<1> Shall I compare thee to a summer's day? Thou art
            --- next slide, please ---
        \item<2-> more lovely, and --- next slide, please ---
        \item<3-> more temperate.
    \end{itemize}
\end{frame}
\end{lstlisting}
\only<3>{Slide 1}
\only<4>{Slide 2}
\only<5>{Slide 3}
\begin{itemize}
    \item<3,6> Shall I compare thee to a summer's day? Thou art
        --- next slide, please ---
    \item<4-> more lovely, and --- next slide, please ---
    \item<5-> more temperate.
\end{itemize}
\onslide<6>\bf{Problem: requires manual work, even for simplest effect}
\end{fragileframe}

\begin{fragileframe}{Incremental specifications}
Author of \tt{beamer} knew this, and came up with
\bf{incremental specifications}:
\if\pre1\pause\fi
\begin{lstlisting}
\begin{frame}
    \begin{itemize}
        \item<+-> ...
        \item<+-> ...
    \end{itemize}
\end{frame}
\end{lstlisting}
\if\pre1\pause\fi
\only<3>{Slide 1}
\only<4>{Slide 2}
\only<5>{Slide 3}
\begin{itemize}
    \item<+-> It is a truth universally acknowledged, that a single man
        in possession of --- next slide, please ---
    \item<+-> a good fortune, must be in want of --- next slide, please ---
    \item<+-> a wife.
\end{itemize}
\onslide<6>\bf{Problem: even this is too much work}
\end{fragileframe}

\begin{fragileframe}{Incremental specifications}
\bf{They knew.}
\if\pre1\pause\fi
The previous example is equivalent to
\begin{lstlisting}
\begin{frame}
    \begin{itemize}[<+->]
        \item ...
        \item ...
    \end{itemize}
\end{frame}
\end{lstlisting}
\end{fragileframe}

\begin{fragileframe}{\tt{<+->} explained}
How does \tt{<+->} work?
\begin{itemize}
    \item Counter called \tt{beamerpauses}
    \item \verb|\pause| adds 1 to \tt{beamerpauses}
    \item Things like \verb|<2->| do not
    \item \verb|+| expands to current value of \tt{beamerpauses}, then adds 1
    \item \verb|<+->| expands to \verb|<1->|, \verb|<2->|, etc
    \item Unless you \verb|\pause|d
\end{itemize}
\if\pre1\pause\fi
\begin{diy}{5}
    Follow instructions in \tt{diy.tex} to create your own overlays.
\end{diy}
\end{fragileframe}

\section{Lifehacks}
\begin{fragileframe}{Aspect ratio}
Beamers are 4:3 by default. Switch to 16:9 with
\begin{lstlisting}
\documentclass[aspectratio=169]{beamer}
\end{lstlisting}
Page size (mm):
\begin{itemize}
    \item 4:3 --- 128$\times$96
    \item 16:9 --- 160$\times$90
\end{itemize}
This means vertical content \textit{might} overflow.
Better settle on an aspect ratio from the beginning.
\end{fragileframe}

\begin{fragileframe}{Formulas}
Works as you would expect in an article, except:
\begin{columns}
    \begin{column}{0.4\textwidth}
        \begin{block}{Consistent}
\begin{lstlisting}
\int x^2 \mathsf{d} x
\end{lstlisting}
            \[
                \int x^2 \mathsf{d} x
            \]
        \end{block}
    \end{column}
    \begin{column}{0.4\textwidth}
        \begin{block}{Not so much}
\begin{lstlisting}
\int x^2 \mathrm{d} x
\end{lstlisting}
            \[
                \int x^2 \mathrm{d} x
            \]
        \end{block}
    \end{column}
\end{columns}
\tt{rm} = \textrm{Roman}; \tt{sf} = \textsf{sans-serif}
\begin{block}{Dirty hack}
    Too many \verb|\mathrm|s to replace? Try
\begin{lstlisting}
\renewcommand{\mathrm}{\mathsf}
\end{lstlisting}
\end{block}
\end{fragileframe}

\begin{fragileframe}{Fragile frames}
If you include code, you have to tell \tt{beamer} that this frame is
``fragile''.
\begin{lstlisting}
\begin{frame}[fragile]
    % Content that contains code
\end{frame}
\end{lstlisting}
\end{fragileframe}

\begin{fragileframe}{Remove navigation symbols}
In the bottom right of your DIY slides are the navigation symbols:
\includegraphics[height=1cm]{nav}

This line removes them:
\begin{lstlisting}
\setbeamertemplate{navigation symbols}{}
\end{lstlisting}

Why?
\begin{itemize}
    \item Distracting clutter
    \item Manuel hates them
    \item Have you ever \textit{used} it?
\end{itemize}
\end{fragileframe}

\begin{frame}{Further reading / References}
\begin{itemize}
    \item The \tt{beamer} user guide:
        \url{http://mirrors.ctan.org/macros/latex/contrib/beamer/doc/beameruserguide.pdf}
    \begin{itemize}
        \item Multicolumn: \S 12.7
        \item Sections: \S 10.2
        \item Blocks: \S 12.3
        \item Overlays: \S 9
    \end{itemize}
    \item \LaTeX\ workshop by Liu Yihao:
        \url{https://github.com/SJTU-UMJI-Tech/LaTeX/blob/master/build/c5_beamer.pdf}
\end{itemize}
\end{frame}

\begin{frame}
\Huge{Thanks!}
\end{frame}
\end{document}
